\documentclass[12pt,a4paper]{article}
\usepackage[T2A]{fontenc}
\usepackage[utf8x]{inputenc}
\usepackage[english,russian]{babel}
\usepackage{amsmath,amsfonts,amssymb,amsthm,mathtools}

\author{Ящер с планеты Цугундер}
\title{\textit{Бе}се\textit{ды} с \textit{Ба}тю\textit{шкой}}
\date{\today}

\begin{document}

\maketitle
\newpage

\section*{Гомеоморфизм}

Пусть $M,N \subset \mathbb{R}^{n}.$ $f:M\longrightarrow N$ -- отображение\footnote{Отображение $f:M\rightarrow N$ -- закон, который каждому элементу $x \in M$ ставит в  соответствие единственный элемент $y \in N.$}.\\


Отображение $f$ \textbf{непрерывно в точке $a$}, тогда и только тогда, когда 
		\[\lim_{x \to a}{f(x)} = f(a)\]

\textbf{Непрерывность по Коши:}
	\[ 
		\forall \varepsilon > 0 ~
		\exists \delta_{\varepsilon} > 0: ~
		(\forall x\in U_{\delta_{\varepsilon}}(a)\cap M
		\Rightarrow f(x)\in U_{\varepsilon}(f(a))) 
	\]
\begin{center}
	или в многомерном случае
\end{center}
	\[
		\forall \varepsilon > 0	~
		\exists \delta_{\varepsilon} > 0:
		(\forall x\in B_{\delta_{\varepsilon}}(a)
		\Rightarrow \rho(f(x),f(a))<\varepsilon),
		\footnote{
			Здесь ~
			$B_{\delta_{\varepsilon}} = 
			\{y\in \mathbb{R}^n | ~ \rho(y,a) <
			\delta_{\varepsilon}\}$,
			а ~
			$\rho(x,y) = \sqrt{\sum_{i=1}^n(x_i - y_i)^2}$
				}
	\]	
	
	
	\textbf{По Гейне:}
		\[ \forall  x_n \to a \Rightarrow f(x_n) \to f(a) \]\\
	
	\textbf{\large{Определение:}} отображение $f:M \to N$ называется \underline{\textbf{гомеоморфизмом}}, если
	\begin{itemize}
		\item $f$ -- биекция,
		\item $f$ -- непрерывна,
		\item $f^{-1}$ -- непрерывна
	\end{itemize}
		









\end{document}

